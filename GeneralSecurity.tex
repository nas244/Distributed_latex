%!TEX root =  root.tex

\section{General Security}
\subsection{CIAA}  
%cite Martin 2017 here
CIAA is one of the main approaches to security classifications utilized by cybersecurity experts and we will be using it for classification purposes. Fully enumerated CIAA stands for confidentiality, integrity, availability, and authentication. This classification system allows security concerns to be broken into four distinct categories to aid in security assessment and discussion. 

\subsubsection{C - Confidentiality}

%change wording, esp. "determines"

Maintaining the confidentiality of data and assets primarily refers to maintaining access control between data and users. Specifically, this idea has to do with ensuring that only select users can access data or assets that they should and is considered breached when non-authorized users can view or gain access. Data includes items such as network packets or other items of a digital nature that the system gather, process, and possibly save. Assets items include hardware and specific software structures that the system is using to process data and complete other tasks. For both of these items confidentiality is highly important as having unauthorized access to either or both creates the possibility for compromised systems and abuse. 

\subsubsection{I - Integrity}

Integrity of data and assets means ensuring that the items maintain a level of ability to detect when modifications have been made and/or prevented. For many systems data integrity is the largest concern as the data that is being processed and having decisions made upon should be as correct as possible and that the system should be able to determine if the data appears to be altered though either data corruption or by a malicious actor. For a robotic system concerns of asset integrity is necessary as well as physical access to the robot is possible in many cases and thus the robot must be able to detect and be confident in it's asset's ability to function within expectations.

\subsubsection{A - Availability}

Maintaining the availability of data and assets means that these resources are accessible to the authorized users when requested. This idea can be at odds with confidentiality because to make the items more available inherently reduces the items confidentiality. Thus it is important to maintain a balance between the two items as the proper users should be able to access the items otherwise the data or asset can be rendered useless.  

\subsubsection{A - Authentication}

This \cite{article:Martin} paper discusses this.

Authentication is the most recently added letter to the CIAA triad but for good reason. In the most broad sense authentication requires that a user prove that they have the proper credentials to access data or assets and can be used as one way to prevent breaches too confidentiality and ensure that the data is available to those that require access. For robots the idea of authentication can be incredibly important as physical access to the robot can be difficult to prevent and thus authentication methods have to be put in place to prevent tampering.
\\\\
Along with authentication, some security theorists include non-repudiation. This is typically defined as any method which can be used to prove that an action was taken by a specific user. The most common implementation of this involves using digital signing, a form of asymmetric cryptography, but this can be computationally expensive. For robotic systems having some form of non-repudiation can be important depending on the tasks attempting to be accomplished but due to its more resource intensive nature, it can be difficult to justify adding it to systems where typically resources are limited. 
\\\\
It is important to note that these properties do not control system design, but rather, these are aspects of good system design. Any secure system will have all of these principles included in some form or another, however by simply being present it does not mean that a system is infallible. Many consideration will have to be made at many levels of a robotic network to build a secure system as many attacks will attempt to break at least one of major aspects of CIAA along a specific attack vector.

\subsection{Goals of Adversary}

The goals of an adversary to a computer system can be widely varied. On a philosophical level, it is to break one or more of the aspects of CIAA, but realistically speaking these goals are typically more practical and concrete. For a mobile robotic network these goals could include things such as intercepting and/or spoofing transmissions, attempting to remove a robot from the network either physically or through software, attempt to take control of a robot, etc. All of these goals are different in their end result, likely attack vectors, and motivations thus can be typically broken into three broad categories of controlling the system, impairment, and data access that will be covered in this paper.
\\  % tie these to multi robot systems? Make logical leaps; do this at the end in one paragraph,  it'll flow better and be less repetitive
\subsubsection{Control of the System}

If an adversary is able to gain control of a robotic system, they are capable of making any changes or view any of the data flowing through the network and typically breaks every aspect of CIAA. Due to the nature of this breach, if successful, the adversary gains full control over every aspect of the system or individual robot and create a large variety of consequences. Since robots are capable of interacting and possibly manipulating the physical environment around them, compromised robots can cause an innumerable number of effects. These can range from destruction of their local environment and possible even themselves but also the possible endangerment of human health and safety. 


\subsubsection{Impairment}

Another possible goal is just to inhibit or prohibit parts or all of the system from functioning. This is different from control over the system as the attacker may only be able to prevent one specific subsystem from functioning as expected instead of being able to change the expected behavior. In terms of the CIAA this typically breaks the availability and possibly integrity aspect.  

\subsubsection{Data Access}

The last area of possible adversarial goals is to gain the ability to intercept and read data being sent through the system. While this is an issue across any computing system, robotic systems in particular carry a larger operational risk in being susceptible to this kind of goal as they have the potential to hold more at-risk data. Examples of this include items such as maps of buildings, operational meta-data about a user's schedule, and other data about the physical environment that can be useful to a malicious user. 

% Link robot specific areas of CIAA


\subsection{Attack Vector}
With the goals established, it is also important to examine the various methods an attack will be carried out on. For the purpose of this paper attack vectors will be broken into two categories, networking attacks and non-networking attacks. For networking attacks the network protocol stack can be used to broadly classify attacks across the five layers which include the physical, data link, network, transport, and application layer. The non-networking attacks will include the categories of physical tampering and system-level exploits.

\subsubsection{Physical Layer}
The physical layer includes electricity on a wire or wave patterns in space. For wireless robotic networks, attacks carried out across the physical layer will involve canceling out or overpowering wireless signals.

\subsubsection{Data Link Layer}
The data link layer is were protocols enter the picture. Ethernet and 802.11 act at this layer. Attacks at this layer will involve spoofing other machines, a form of denial of service (DOS), or jamming by taking advantage of the 802.11 protocol's faults.

\subsubsection{Network Layer}
The Internet Protocol (IP) is the dominant protocol of the network layer. Some examples of this attack at this layer could be DOS, spoofing, or malicious rerouting. Typically attacks at this layer and the following are not specific to robotic networks, but still stand as a threat to them. 

\subsubsection{Transport Layer}
TCP and UDP are the predominant protocols at this layer. Operating system level attacks occur at this layer. Usually, attacks at this layer occur because of poor or nonexistent key management/security.

\subsubsection{Application Layer}
While this is a layer on the network protocol stack most attacks focused on this layer are covered under system-level exploits. This is because while a majority of these attacks are carried out over a network the actual fault being exploited is typically within the application itself.

